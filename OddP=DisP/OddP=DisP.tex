%This template is based on one provided by the American Physical Society for submission to its journals.

\documentclass[aps,onecolumn,superscriptaddress,preprintnumbers]{revtex4-2}

%The following packages add LaTeX commands that make formatting and writing math easier

\usepackage{graphicx}  % Include figure files
\usepackage{subfigure}
\usepackage{multirow}
\usepackage{tikz-feynman}
\usepackage{hyperref}
    
\linespread{1.1}
\usepackage{fancyhdr}
\usepackage{longtable}
\usepackage{parskip}
\usepackage[T1]{fontenc}
\usepackage{dcolumn}   % Align table columns on decimal point

\usepackage{bm}        % bold math
\usepackage{amsfonts}  % Common math fonts
\usepackage{amsmath}   % Common math functions
\usepackage{amssymb}   % Common math symbols

\setlength{\parindent}{0pt}
\newenvironment{rcases}
{\left.\begin{aligned}}
{\end{aligned}\right\rbrace}
  
\newcommand{\qed}{\tag*{$\square$}}

\begin{document}


Integer partitioning function $p(n)$ can be represented as the generating function of the sequence of $p(n)$:
\begin{equation}
\sum_{n=0}^{\infty}p(n)x^n = \prod_{k=1}^{\infty}\frac{1}{1-x^k}.
\end{equation}

The generating function of odd partition, denoted by $p_o(n)$, can be derived as following:
\begin{align*}
\prod_{k=1}^{\infty}\frac{1}{1-x^k} &= \cfrac{1}{1-x} \cdot \cfrac{1}{1-x^2} \cdot \cdots \cdot \cfrac{1}{1-x^m} \cdot \cdots\\
&= \prod_{k=1}^{\infty}(x^0+x^k+x^{2k}+\cdots)\\
&= (1+x+x^2+\cdots )\\
&\hspace{13pt}(1+x^2+x^4+\cdots )\\
&\hspace{13pt}\cdots\\
&\hspace{13pt}(1+x^m+x^{2m}+\cdots )\\
&\hspace{13pt}\cdots
\end{align*}
The correspondence can be represented by using a simple example of $p(5)$.
\begin{alignat*}{3}
5&=1+1+1+1+1&&~\Longrightarrow~ (x^1)^5 \text{~from~} \frac{1}{1-x}\\
&=2+1+1+1&&~\Longrightarrow~ (x^2)^1 \text{~from~} \frac{1}{1-x^2}; ~&&(x^1)^3 \text{~from~} \frac{1}{1-x}\\
&=2+2+1&&~\Longrightarrow~ (x^2)^2 \text{~from~} \frac{1}{1-x^2}; &&(x^1)^1 \text{~from~} \frac{1}{1-x}\\
&=3+1+1&&~\Longrightarrow~ (x^3)^1 \text{~from~} \frac{1}{1-x^3}; &&(x^1)^2 \text{~from~} \frac{1}{1-x}\\
&=3+2&&~\Longrightarrow~ (x^3)^1 \text{~from~} \frac{1}{1-x^3}; &&(x^2)^1\text{~from~} \frac{1}{1-x^2}\\
&=4+1&&~\Longrightarrow~ (x^4)^1 \text{~from~} \frac{1}{1-x^4}; &&(x^1)^1 \text{~from~} \frac{1}{1-x}\\
&=5 &&~\Longrightarrow~ (x^5)^1 \text{~from~} \frac{1}{1-x^5}.
\end{alignat*}
To express the partition function of the odd partition, we need to eliminate partitions that have even numbers in it, which means we won't count it into the odd partition if $a$ in $(x^a)^b$ is even. Therefore, the generating function of the odd partition is
\begin{align}
\sum_{n=0}^{\infty}p_o(n)x^n &= \cfrac{1}{1-x}\cdot\cfrac{1}{1-x^3}\cdot\cfrac{1}{1-x^5}\cdot\cdots\notag\\
&=\prod_{k=0}^{\infty}\frac{1}{1-x^{2n+1}}.
\end{align}
Similarly, we won't count another to the distinct partition if $b$ in $(x^a)^b$ is larger than $1$. Therefore, the expansions will only have their first two terms, and the generating function of the distinct partition is 
\begin{align}
\sum_{n=0}^{\infty}p_d(n)x^n &= (1+x+0)\cdot(1+x^2+0)\cdot\cdots\notag\\
&=\cfrac{1-x^2}{1-x}\cdot\cfrac{1-x^4}{1-x^2}\cdot\cfrac{1-x^6}{1-x^3}\cdot\cfrac{1-x^8}{1-x^4}\cdot\cdots\\
&=\cfrac{1}{1-x}\cdot\cfrac{1}{1-x^3}\cdot\cfrac{1}{1-x^5}\cdot\cdots.\notag
\end{align}
Therefore, it is proved that since the generating functions are the same, $p_o(n)=p_d(n)$.

\end{document}
