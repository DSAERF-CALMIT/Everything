\documentclass[12pt]{scrartcl}
\title{Moment of Inertia}
\nonstopmode
\usepackage{graphicx}	% Required for including pictures
\usepackage[figurename=Figure]{caption}
\usepackage{float}    	% For tables and other floats
\usepackage{amsmath}  	% For math
\usepackage{bbm}  	% For mathBBM
\usepackage{amssymb}  	% For more math
\usepackage{fullpage} 	% Set margins and place page numbers at bottom center
\usepackage{paralist} 	% paragraph spacing
\usepackage{listings} 	% For source code
\usepackage{enumitem} 	% useful for itemization
\usepackage{siunitx}  	% standardization of si units
\usepackage{tikz,bm} 	% Useful for drawing plots
\usepackage{fancyhdr}
\usepackage{setspace}
\usepackage{mathtools}
\usepackage{hyperref}

\usepackage{amsthm}
\usepackage[framemethod=TikZ]{mdframed}
%%%%%%%%%%%%%%%%%%%%%%%%%%%%%%
%Proof
\newcounter{prf}[section]\setcounter{prf}{0}
\newenvironment{prf}[2][]{%
\refstepcounter{prf}%
\ifstrempty{#1}%
{\mdfsetup{%
frametitle={%
\tikz[baseline=(current bounding box.east),outer sep=0pt]
\node[anchor=east,rectangle,fill=red!20]
{\strut \textit{proof}};}}
}{}%
\mdfsetup{innertopmargin=5pt,linecolor=red!25,%
linewidth=1.5pt,topline=true,%
frametitleaboveskip=\dimexpr-\ht\strutbox\relax
}
\begin{mdframed}[]\relax%
\label{#2}}{\end{mdframed}}


\hypersetup{
    colorlinks=true,
    linkcolor=violet,
    urlcolor=violet,
}


\setstretch{1.2}

\renewcommand{\headrulewidth}{0pt}
\renewcommand\footrule{\hrule height1pt}

\pagestyle{fancy}
\fancyhf{}
\lfoot{Moment of Inertia \raisebox{0.5\depth}{\scalebox{0.8}\textcopyright}~Jinwei Zou}
\rfoot{Page \thepage}

\begin{document}

	\begin{center}
		\textbf{ \large Moment of Inertia \raisebox{0.5\depth}{\scalebox{0.8}\textcopyright}~BinaryPhi}
	\end{center}
	\vspace{-0.3em}
	\hspace{\fill} \textbf{Last Edit:} \today~PDT \vspace{-0.6em} \\
	\hrule

\vspace{3em}

%%%%%%%%%%%%%%%%
{\fontfamily{lmss}\selectfont
	\noindent \textbf{\large Definitions:}
}
\vspace{1em}

	\noindent \textbf{Moment of Inertia}, often denoted by $I$ (SI Unit: $\text{kg}\cdot\text{m}^2$), is a quantitative measurement of the inertia for an object to rotate about a specific axis. It can be calculated by the summation of moment of inertia of every infinitesimal part of the rigid body ($\Longrightarrow$ in other words, a \textbf{mass point}). The moment of inertia of a mass point is:
	\begin{equation} \label{eq1}
	I = m \hspace{2pt} r^2,
	\end{equation}

\noindent where $m$ is the mass of the mass point and $r$ is the distance between this mass point and the rotational axis. By using summation, the moment of inertia of a system can be written as:
	\begin{equation} \label{eq2}
	I = \sum_i^{\infty} m_i \hspace{2pt} r_i^2.
	\end{equation}
\newline
\noindent By using integral, the moment of inertia can be written as:
	\begin{equation} \label{eq3}
	I = \int r^2 dM.
	\end{equation}	
\newline
\noindent Depending on the circumstances, we can transform \eqref{eq3} to the following:
	\begin{align} 
	I &= \int \rho r^2 dV; \label{eq4} \\
	I &= \int \sigma r^2 dS; \label{eq5} \\
	I &= \int \lambda r^2 dL. \label{eq6}
	\end{align}
\newline
\noindent In \eqref{eq4} \eqref{eq5} \eqref{eq6}, the convention is $\rho$ denotes the Volume Density, $\sigma$ denotes the Area Density, and $\lambda$ denotes the Linear Density. In addtion, 
	$$dL = Rd\theta$$
is also useful in some cases.

%Additionally, we could use a rank-2 tensor: 
%	$$\begin{pmatrix}
%	I_{xx} & I_{xy} & I_{xz}\\
%	I_{yx} & I_{yy} & I_{yz}\\
%	I_{zx} & I_{zy} & I_{zz}
%	\end{pmatrix}$$

\newpage
%%%%%%%%%%%%%%%%
{\fontfamily{lmss}\selectfont
	\noindent \textbf{Infinite Thin Rod $:$ Center }(Mass: \textbf{m}; Lenght: \textbf{l})
}
\begin{equation}
I = \frac{1}{12}ml^2.
\end{equation}
\vspace{-2.2em}
\begin{prf}{prf:proof1}
\vspace{-2em}
\addtolength{\jot}{0.4em}
\begin{align*}
I &= \int r^2 dM = \int_0^{m} r^2 dM \\
&= \int r^2 (\lambda dL) = \int_{-l/2}^{l/2} \frac{m}{l} L^2 dL \\
&= \frac{m}{l} \int_{-l/2}^{l/2} L^2 dL = \frac{m}{l} \cdot \large\left[ \frac{1}{3}L^{3} \large\right]_{-l/2}^{l/2} \\
&= \frac{m}{l} \left[\frac{1}{3}\left(\frac{l}{2}\right)^{3} - \frac{1}{3}\left(-\frac{l}{2}\right)^{3}\right] \\
\Aboxed{&\hspace{3pt}= \frac{1}{12}ml^2.~}\\
\end{align*}
\vspace{-3em}
\end{prf}

\vspace{2em}
%%%%%%%%%%%%%%%%
\setlength\leftskip{0pt}{\fontfamily{lmss}\selectfont
	\noindent \textbf{Infinite Thin Rod $:$ One End }(Mass: \textbf{m}; Lenght: \textbf{l})
}
\begin{equation}
I = \frac{1}{3}ml^2.
\end{equation}
\vspace{-2.2em}
\begin{prf}{prf:proof2}
\vspace{-2em}
\addtolength{\jot}{0.4em}
\begin{align*}
I &= \int r^2 dM = \int_0^{m} r^2 dM \\
&= \int r^2 (\lambda dL) = \int_{0}^{l} \frac{m}{l} L^2 dL \\
&= \frac{m}{l} \int_{0}^{l} L^2 dL \\
&= \frac{m}{l} \cdot \large\left[ \frac{1}{3}L^{3} \large\right]_{0}^{l} = \frac{m}{l} \left(\frac{1}{3}l^{3} - 0\right) \\
\Aboxed{&\hspace{3pt}= \frac{1}{3}ml^2.~}\\
\end{align*}
\vspace{-3em}
\end{prf}


%%%%%%%%%%%%%%%%
\setlength\leftskip{0pt}{\fontfamily{lmss}\selectfont
	\noindent \textbf{Parallel Axis Theorem}
}

\vspace{1.5em}
\noindent The moment of inertia of a rigid body about a rotating axis is: 
\begin{equation}
I = I_{cm} + md^2.
\end{equation}
$I_{cm}$ is the moment of inertia when rotating about the axis that passes through the center of mass. $d$ is the distance between an arbitrary rotational axis and the axis that passes through the center of mass of the object.
\begin{prf}{prf:proof3}
\addtolength{\jot}{0.5em}
For the sake of simplicity, let's imagine the rigid body rotating about the $z$-axis, which passes through the center of mass of the body. You could use tensor to do a more generalized proof (I will do a tensor version of this whole topic if I have time).
\begin{align*}
I_{cm} &= \int r^2 dM\\
&= \int (x^2+y^2+z^2) dM = \int (x^2+y^2) dM\\
I &= \int \left((x-d_1)^2+(y-d_2)^2\right) dM\\
&= \int \left( x^2 + y^2 - 2d_1x - 2d_2y + d_1^2 + d_2^2 \right)dM\\
\begin{split}
~\hspace{-4pt}= \int (x^2+y^2) dM + (d_1^2+d_2^2) \int dM& \\
- ~ 2d_1 \int x dM - 2d_2 \int y dM& \\
\end{split}
\\ \because \hspace{3pt}~&~~~\hspace{2pt} \int x dM = \int y dM = 0\\
\therefore I &= \int (x^2+y^2) dM + d^2 \cdot \int dM\\
\Aboxed{&\hspace{3pt}= I_{cm} + md^2}\\
\end{align*}
\vspace{-3em}
\end{prf}

\newpage
%%%%%%%%%%%%%%%%
\setlength\leftskip{0pt}{\fontfamily{lmss}\selectfont
	\noindent \textbf{Circular Thin Loop $:$ Orthogonal Center} (Mass: \textbf{m}; Radius: \textbf{R})
}
\begin{equation}
I_z = mR^2.
\end{equation}
\vspace{-2.3em}
\begin{prf}{prf:proof2}
\vspace{-2em}
\addtolength{\jot}{0.5em}
\begin{align*}
I_z &= \int r^2 dM = \int R^2 dM\\
&= R^2 \int dM = R^2 \cdot m\\
\Aboxed{&\hspace{3pt}= mR^2}\\
\end{align*}
\vspace{-3.5em}
\end{prf}

\vspace{1.5em}
%%%%%%%%%%%%%%%%
\setlength\leftskip{0pt}{\fontfamily{lmss}\selectfont
	\noindent \textbf{Circular Thin Loop $:$ Diameter} (Mass: \textbf{m}; Radius: \textbf{R})
}
\begin{equation}
I_x = I_y = \frac{1}{2}mR^2.
\end{equation}
\vspace{-2.3em}
\begin{prf}{prf:proof2}
\vspace{-2em}
\addtolength{\jot}{0.5em}
\begin{align*}
I_x  = I_y &= \int r^2 dM\\
&= \int r^2 (\lambda dL) = \int_0^{2\pi} r^2 \left( \frac{m}{2\pi R} R d\theta\right) \\
&= \int_0^{2\pi} (R \text{cos}\theta)^2 \left( \frac{m}{2\pi} \right) d\theta\\
&= R^2\frac{m}{2\pi} \int_0^{2\pi}  \text{cos}^2\theta d\theta\\
&= R^2\frac{m}{2\pi} \int_0^{2\pi}  \frac{1+\text{cos}2\theta}{2} d\theta\\
&= R^2\frac{m}{2\pi} \left( \left[ \frac{1}{2} \theta + \frac{1}{4}\text{sin}2\theta \large\right]_0^{2\pi} \right)\\
&= R^2\frac{m}{2\pi} \left( \pi - 0 + 0 - 0 \right)\\
\Aboxed{&\hspace{3pt}= \frac{1}{2}mR^2.}\\
\end{align*}
\vspace{-3.5em}
\end{prf}




\end{document}















